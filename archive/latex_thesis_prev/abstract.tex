\thispagestyle{plain}
\begin{center}
    \Large
    \textbf{Detecting emotion in text}
    
    \vspace{0.4cm}
    \large
    Using Facebook reactions
    
    \vspace{0.4cm}
    \textbf{Chris Pool}
    
    \vspace{0.9cm}
    \textbf{Abstract}
\end{center}
Sentiment analysis is a term used for many different tasks but are all related. Most commonly this term is used for the task of detecting the polarity or valence in a piece of text, in most cases \texttt{positive, neutral or negative}. Some other authors use the term for a broader task, detecting emotions i.e.( \texttt{anger, joy, fear, surprise, sadness and disgust}) and feelings in text. Both tasks are increasing popular research areas due to the fact that increasing amounts of data is available via social media and this information can be useful for understanding how people feel about a certain topic, or what mood the user is in.\\\\
In this thesis I explored the possibility to use data from Facebook as data to train a model that can classify to dominant emotion in a piece of text. Facebook enables users to respond to a post with an emotion, this emotion feature was introduced in February 2016 and was an addition to the well-known like button. Facebook introduced it so people had more options to quickly respond to a post.\\\\
I collected posts and the reaction of users about these posts and used the most used reaction as a label for that post. I trained a SVM model using several features, the most important feature where the words that were used. I evaluated my final system on three data-sets commonly used in emotion classification. While it performed better then most unsupervised systems it performs slightly less then supervised systems. The difference between supervised systems and my system differ regarding the domain of the test-set.\\\\





\clearpage